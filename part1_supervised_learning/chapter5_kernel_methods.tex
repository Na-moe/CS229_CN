\chapter{核方法}

\section{特征映射}

回顾在线性回归的讨论中,考虑了根据房屋的居住面积(记为 $x$)预测房屋价格(记为 $y$)的问题,并将 $x$ 的线性函数拟合到训练数据。如果价格 $y$ 可以更准确地表示为 $x$ 的\textit{非线性 (non-linear)}函数呢?在这种情况下,需要一个比线性模型更具表现力的模型族。

首先考虑拟合三次函数 $y = \theta_3 x^3 + \theta_2 x^2 + \theta_1 x + \theta_0$。结果表明,可以将三次函数视为在不同特征变量集(定义如下)上的线性函数。具体来说,令函数 $\phi: \mathbb{R} \to \mathbb{R}^4$ 定义为
\begin{equation}
    \phi(x) = \begin{bmatrix} &1& \\ &x& \\ &x^2& \\ &x^3& \end{bmatrix} \in \mathbb{R}^4.
    \label{eq:kernel_phi}
\end{equation}

令 $\theta \in \mathbb{R}^4$ 是包含 $\theta_0, \theta_1, \theta_2, \theta_3$ 作为元素的向量。那么可以将三次函数写成 $x$ 的形式:
\[
    \theta_3 x^3 + \theta_2 x^2 + \theta_1 x + \theta_0 = \theta^T \phi(x).
\]
因此,变量 $x$ 的三次函数可以视为变量 $\phi(x)$ 上的线性函数。为了区分这两组变量,在核方法的背景下,将问题的“原始”输入值称为输入\textbf{属性 (attributes)}(在本例中为 $x$,即居住面积)。当原始输入被映射到一组新的量 $\phi(x)$ 时,将这些新的量称为\textbf{特征 (features)} 变量。(不幸的是,不同的作者在不同的语境下使用不同的术语来描述这两者。)将 $\phi$ 称为\textbf{特征映射 (feature map)},它将属性映射到特征。

\section{带特征的最小均方}

现在我们推导拟合模型 $\theta^T \phi(x)$ 的梯度下降算法。首先回顾一下,对于普通的最小二乘问题,拟合 $\theta^T x$ 的批量梯度下降更新(其推导参见讲义第一章)为:

\begin{align} 
    \theta &:= \theta + \alpha \sum_{i=1}^n (y^{(i)} - h_\theta(x^{(i)})) x^{(i)} \notag\\ 
    &:= \theta + \alpha \sum_{i=1}^n (y^{(i)} - \theta^T x^{(i)}) x^{(i)}.\label{eq:kernel_ols}
\end{align}

令 $\phi: \mathbb{R}^d \to \mathbb{R}^p$ 是一个将属性 $x$(在 $\mathbb{R}^d$ 中)映射到 $\mathbb{R}^p$ 中的特征 $\phi(x)$ 的特征映射。(在前面小节的示例中,$d=1$ 且 $p=4$。)现在目标是拟合函数 $\theta^T \phi(x)$,其中 $\theta$ 是 $\mathbb{R}^p$ 中的向量而不是 $\mathbb{R}^d$ 中的向量。可以将上面算法中 $x^{(i)}$ 的所有出现替换为 $\phi(x^{(i)})$,得到新的更新:
\begin{equation}
    \theta := \theta + \alpha \sum_{i=1}^n (y^{(i)} - \theta^T \phi(x^{(i)})) \phi(x^{(i)}).
    \label{eq:kernel_iterate}
\end{equation}
类似地,相应的随机梯度下降更新规则是
\begin{equation}
    \theta := \theta + \alpha (y^{(i)} - \theta^T \phi(x^{(i)})) \phi(x^{(i)}).
\end{equation}

\section{带核技巧的最小均方}

当特征 $\phi(x)$ 是高维时,上面的梯度下降或随机梯度下降更新在计算上变得昂贵。例如,考虑将方程 \eqref{eq:kernel_phi} 中的特征映射直接扩展到高维输入 $x$:假设 $x \in \mathbb{R}^d$,令 $\phi(x)$ 是包含所有 $x$ 的次数 $\le 3$ 的项的向量
\begin{equation}
    \phi(x) = \begin{bmatrix} &1& \\ &x_1& \\ &x_2& \\ &\vdots& \\ &x_1^2& \\ &x_1 x_2& \\ &x_1 x_3& \\ &\vdots& \\ &x_2 x_1& \\ &\vdots& \\ &x_1^3& \\ &x_1^2 x_2& \\ &\vdots& \\ &x_2 x_3 x_1& \\ &\vdots& \end{bmatrix}. \label{eq:5.5}
\end{equation}
特征 $\phi(x)$ 的维度大约是 $d^3$ 量级\footnote{此处,为简单起见,包含所有重复的单项式(因此,例如 $x_1 x_2 x_3$ 和 $x_2 x_3 x_1$ 都出现在 $\phi(x)$ 中)。因此,$\phi(x)$ 中共有 $1 + d + d^2 + d^3$ 个元素。}。对于计算来说,这是一个非常长的向量——当 $d = 1000$ 时,每次更新需要至少计算和存储一个 $1000^3 = 10^9$ 维向量,这比普通最小二乘更新规则 \eqref{eq:kernel_ols} 慢 $10^6$ 倍。

乍一看,每次更新 $d^3$ 的运行时长和内存使用似乎是不可避免的,因为向量 $\theta$ 本身的维度是 $p \approx d^3$,并且可能需要更新和存储 $\theta$ 的每一个元素。然而,将引入核技巧,通过它不需要显式存储 $\theta$,并且运行时长可以显著改善。

为简单起见,假设初始值 $\theta = 0$,并且关注迭代更新 \eqref{eq:kernel_iterate}。主要观察是,在任何时候,$\theta$ 都可以表示为向量 $\phi(x^{(1)}), \dots, \phi(x^{(n)})$ 的线性组合。实际上,可以通过归纳法证明如下。在初始化时,$\theta = 0 = \sum_{i=1}^n 0 \cdot \phi(x^{(i)})$。假设在某个时刻,$\theta$ 可以表示为
\begin{equation}
    \theta = \sum_{i=1}^n \beta_i \phi(x^{(i)}).
\end{equation}
其中 $\beta_1, \dots, \beta_n \in \mathbb{R}$。然后,断言在下一轮中,$\theta$ 仍然是 $\phi(x^{(1)}), \dots, \phi(x^{(n)})$ 的线性组合,因为
\begin{align} 
    \theta &:= \theta + \alpha \sum_{i=1}^n (y^{(i)} - \theta^T \phi(x^{(i)})) \phi(x^{(i)}) \notag\\ 
    &= \sum_{i=1}^n \beta_i \phi(x^{(i)}) + \alpha \sum_{i=1}^n (y^{(i)} - \theta^T \phi(x^{(i)})) \phi(x^{(i)}) \notag\\ 
    &= \sum_{i=1}^n \underbrace{(\beta_i + \alpha (y^{(i)} - \theta^T \phi(x^{(i)})))}_{\text{新的}\  \beta_i} \phi(x^{(i)}). 
\end{align}
可以意识到,一般策略是通过一组系数 $\beta_1, \dots, \beta_n$ 隐式表示 $p$ 维向量 $\theta$。为了做到这一点,推导系数 $\beta_i$ 的更新规则。使用上面的方程,可以看到新的 $\beta_i$ 依赖于旧的 $\beta_i$
\begin{equation}
    \beta_i := \beta_i + \alpha (y^{(i)} - \theta^T \phi(x^{(i)})).
\end{equation}
这里,方程的右侧仍然有旧的 $\theta$。将 $\theta$ 替换为 $\theta = \sum_{j=1}^n \beta_j \phi(x^{(j)})$ 得到
\[
    \forall i \in \{1, \dots, n\}, \beta_i := \beta_i + \alpha \left( y^{(i)} - \sum_{j=1}^n \beta_j \phi(x^{(j)})^T \phi(x^{(i)}) \right).
\]
通常将 $\phi(x^{(j)})^T \phi(x^{(i)})$ 重写为 $\langle \phi(x^{(j)}), \phi(x^{(i)}) \rangle$ 以强调它是两个特征向量的内积。通过将 $\beta_i$ 视为 $\theta$ 的新表示,已经成功地将批梯度下降算法转化为一个迭代更新 $\beta$ 值的新算法。它可能看起来在每次迭代时,仍然需要计算所有 $i, j$ 对的 $\langle \phi(x^{(j)}), \phi(x^{(i)}) \rangle$ 的值,其中每个计算可能需要大约 $O(p)$ 的操作。然而,有两个重要的特性可以解决这个问题:
\begin{enumerate}
    \item 在循环开始之前,可以预先计算所有 $i, j$ 对的成对内积 $\langle \phi(x^{(j)}), \phi(x^{(i)}) \rangle$。
    \item 对于定义在 \eqref{eq:5.5} 中的特征映射 $\phi$(或许多其他有趣的特征映射),计算 $\langle \phi(x^{(j)}), \phi(x^{(i)}) \rangle$ 可以是高效的并且不必需显式计算 $\phi(x^{(i)})$。这是因为:
    \begin{align} 
        \langle \phi(x), \phi(z) \rangle &= 1 + \sum_{i=1}^d x_i z_i + \sum_{i,j \in \{1, \dots, d\}} x_i x_j z_i z_j + \sum_{i,j,k \in \{1, \dots, d\}} x_i x_j x_k z_i z_j z_k \notag\\ 
        &= 1 + \sum_{i=1}^d x_i z_i + \left( \sum_{i=1}^d x_i z_i \right)^2 + \left( \sum_{i=1}^d x_i z_i \right)^3 \notag\\ 
        &= 1 + \langle x, z \rangle + \langle x, z \rangle^2 + \langle x, z \rangle^3 \label{eq:5.9}
    \end{align}
    因此,要计算 $\langle \phi(x), \phi(z) \rangle$,可以首先用 $O(d)$ 的时间计算 $\langle x, z \rangle$,然后进行一些常数次操作来计算 $1 + \langle x, z \rangle + \langle x, z \rangle^2 + \langle x, z \rangle^3$。
\end{enumerate}

正如将看到的,特征 $\phi(x), \phi(z)$ 之间的内积在这里至关重要。将与特征映射 $\phi$ 对应的\textbf{核 (Kernel)}定义为一个 $\mathcal{X} \times \mathcal{X} \to \mathbb{R}$ 的函数,满足\footnote{回想一下, $\mathcal{X}$ 是输入 $x$ 的取值空间。在当前示例中,$\mathcal{X} = \mathbb{R}^d$。}:
\begin{equation}
    K(x, z) \triangleq \langle \phi(x), \phi(z) \rangle
\end{equation}

最终算法总结如下:

\noindent\hrulefill
\begin{enumerate}
    \item 使用方程 \eqref{eq:5.9} 计算所有 $i, j \in \{1, \dots, n\}$ 的值 $K(x^{(i)}, x^{(j)}) \triangleq \langle \phi(x^{(i)}), \phi(x^{(j)}) \rangle$。设置 $\beta := 0$。
    \item \textbf{循环}:
    \begin{equation}
        \forall i \in \{1, \dots, n\}, \beta_i := \beta_i + \alpha \left( y^{(i)} - \sum_{j=1}^n \beta_j K(x^{(i)}, x^{(j)}) \right).
        \label{eq:kernel_algo}
    \end{equation}
    或者用向量表示法,令 $K$ 是一个 $n \times n$ 矩阵,其中 $K_{ij} = K(x^{(i)}, x^{(j)})$,有
    \[
        \beta := \beta + \alpha (\vec{y} - K \beta)
    \]
\end{enumerate}
\hrulefill

通过上面的算法,可以在每次更新时以 $O(n)$ 的时间高效地更新向量 $\theta$ 的表示 $\beta$。最后,需要证明表示 $\beta$ 的知识足以计算预测 $\theta^T \phi(x)$。实际上,有
\begin{equation}
    \theta^T \phi(x) = \sum_{i=1}^n \beta_i \phi(x^{(i)})^T \phi(x) = \sum_{i=1}^n \beta_i K(x^{(i)}, x).
    \label{eq:kernel_essence}
\end{equation}
可以意识到,本质上所有需要知道的关于特征映射 $\phi(\cdot)$ 的信息都封装在相应的核函数 $K(\cdot, \cdot)$ 中。将在下一节中对此进行详细阐述。

\section{核的性质}

在上一小节中,从一个显式定义的特征映射 $\phi$ 开始,导出了核函数 $K(x, z) \triangleq \langle \phi(x), \phi(z) \rangle$。然后,看到核函数是如此本质,只要核函数被定义,整个训练算法就可以完全用核方法的语言编写,而无需引用特征映射 $\phi$,因此对于测试示例 $x$ 的预测(方程 \eqref{eq:kernel_essence})也是如此。

因此,可以尝试定义其他核函数 $K(\cdot, \cdot)$ 并运行算法 \eqref{eq:kernel_algo}。请注意,算法 \eqref{eq:kernel_algo} 不需要显式访问特征映射 $\phi$,因此只需要确保特征映射 $\phi$ 的存在,但不一定需要能够显式写下 $\phi$。

哪些类型的函数 $K(\cdot, \cdot)$ 可以对应于某个特征映射 $\phi$? 换句话说,如果存在某个特征映射 $\phi$ 使得对于所有 $x, z$ 都有 $K(x, z) = \phi(x)^T \phi(z)$,能否判断出来?

如果可以通过给出有效核函数的精确表征来回答这个问题,那么就可以完全改变选择核函数 $K$ 的接口,而不是选择特征映射 $\phi$ 的接口。具体来说,可以选取一个函数 $K$,验证它满足该表征(从而存在一个与 $K$ 对应的特征映射 $\phi$),然后就可以运行更新规则 \eqref{eq:kernel_algo}。这里的好处是,不需要能够计算或解析地写下 $\phi$,只需要知道它的存在性。在本小节的末尾,在通过几个具体的核示例之后,将回答这个问题。

假设 $x, z \in \mathbb{R}^d$,首先考虑函数 $K(\cdot, \cdot)$ 定义为:
\[
    K(x, z) = (x^T z)^2.
\]
也可以将其写为
\begin{align} 
    K(x, z) &= \left( \sum_{i=1}^d x_i z_i \right) \left( \sum_{j=1}^d x_j z_j \right) \\ &= \sum_{i=1}^d \sum_{j=1}^d x_i x_j z_i z_j \\ &= \sum_{i,j=1}^d (x_i x_j)(z_i z_j) 
\end{align}
因此,可以看到 $K(x, z) = \langle \phi(x), \phi(z) \rangle$ 是与特征映射 $\phi$ 对应的核函数(这里以 $d=3$ 的情况为例)由下式给出
\[
    \phi(x) = \begin{bmatrix}
        &x_1 x_1& \\
        &x_1 x_2& \\
        &x_1 x_3& \\
        &x_2 x_1& \\
        &x_2 x_2& \\
        &x_2 x_3& \\
        &x_3 x_1& \\
        &x_3 x_2& \\
        &x_3 x_3&
    \end{bmatrix}.
\]
请回想一下核的计算效率,注意,虽然计算高维的 $\phi(x)$ 需要 $O(d^2)$ 的时间,但找到 $K(x, z)$ 只需 $O(d)$ 的时间——与输入属性的维度呈线性关系。

对于另一个相关的例子,也考虑由下式定义的 $K(\cdot, \cdot)$:
\[
    K(x, z) = (x^T z + c)^2
\]
\[
    = \sum_{i,j=1}^d (x_i x_j)(z_i z_j) + \sum_{i=1}^d (\sqrt{2c} x_i)(\sqrt{2c} z_i) + c^2.
\]
(请自行验证)这个函数 $K$ 是一个核函数,它对应到特征映射(再次以 $d=3$ 为例)
\[
    \phi(x) = \begin{bmatrix}
        &x_1 x_1& \\
        &x_1 x_2& \\
        &x_1 x_3& \\
        &x_2 x_1& \\
        &x_2 x_2& \\
        &x_2 x_3& \\
        &x_3 x_1& \\
        &x_3 x_2& \\
        &x_3 x_3& \\
        &\sqrt{2c} x_1& \\
        &\sqrt{2c} x_2& \\
        &\sqrt{2c} x_3& \\
        &c&
    \end{bmatrix},
\]
其中参数 $c$ 控制 $x_i$(一阶)项和 $x_i x_j$(二阶)项之间的相对权重。

更广泛地说,核 $K(x, z) = (x^T z + c)^k$ 对应于一个特征空间,包含所有次数不超过 $k$ 的一元多项式 $x_{i_1} x_{i_2} \dots x_{i_k}$。然而尽管在这个 $O(d^k)$ 维的高维空间中工作,计算 $K(x, z)$ 仍然只需要 $O(d)$ 的时间,因此即使在如此高维的特征空间中,也永远不需要显式表示出特征向量。

\subsection*{核作为相似性度量}

现在,讨论一下核的另一种视角。直观地(尽管这种直觉有一些问题,但先忽略),如果 $\phi(x)$ 和 $\phi(z)$ 彼此接近,那么可以期望 $K(x, z) = \phi(x)^T \phi(z)$ 很大。反过来,如果 $\phi(x)$ 和 $\phi(z)$ 相距很远——例如,几乎相互正交——那么 $K(x, z) = \phi(x)^T \phi(z)$ 将很小。因此,可以将 $K(x, z)$ 看作是 $\phi(x)$ 和 $\phi(z)$ 之间相似性的一种度量,或者说是 $x$ 和 $z$ 之间相似性的一种度量。

有了这种直觉,假设对于正在研究的某个学习问题,想出了一个函数 $K(x, z)$,认为它可能是衡量 $x$ 和 $z$ 相似性的合理度量。例如,可能选择了
\[
    K(x, z) = \exp \left( -\frac{\|x - z\|^2}{2\sigma^2} \right).
\]
这是衡量 $x$ 和 $z$ 相似性的合理度量,当 $x$ 和 $z$ 接近时接近 1,当 $x$ 和 $z$ 相距很远时接近 0。是否存在一个特征映射 $\phi$ 使得上面定义的核 $K$ 满足 $K(x, z) = \phi(x)^T \phi(z)$? 在这个特殊的例子中,答案是肯定的。这个核被称为\textbf{高斯核 (Gaussian kernel)},并且对应于一个无限维的特征映射 $\phi$。下面,我们将精确地描述一个函数 $K$ 需要满足哪些性质才能成为一个有效的核函数,即对应于某个特征映射 $\phi$。

\subsection*{有效核的必要条件}

现在假设 $K$ 确实是一个有效的核,对应于某个特征映射 $\phi$,首先看看它满足哪些性质。考虑一个包含 $n$ 个点(不一定是训练集)的有限集合 $\{x^{(1)}, \dots, x^{(n)}\}$,并定义一个 $n \times n$ 的方阵 $K$,其 $(i, j)$ 项由 $K_{ij} = K(x^{(i)}, x^{(j)})$ 给出。这个矩阵称为\textbf{核矩阵 (kernel matrix)}。请注意,我们为了方便使用了 $K$ 来表示核函数 $K(x, z)$ 和核矩阵 $K$,因为它们之间有明显的密切关系。

如果 $K$ 是一个有效的核,那么有 $K_{ij} = K(x^{(i)}, x^{(j)}) = \phi(x^{(i)})^T \phi(x^{(j)}) = \phi(x^{(j)})^T \phi(x^{(i)}) = K(x^{(j)}, x^{(i)}) = K_{ji}$,因此 $K$ 必须是对称的。此外,令 $\phi_k(x)$ 表示向量 $\phi(x)$ 的第 $k$ 个坐标,对于任意向量 $z$,有
\begin{align*} 
    z^T K z &= \sum_i \sum_j z_i K_{ij} z_j \\ 
    &= \sum_i \sum_j z_i \phi(x^{(i)})^T \phi(x^{(j)}) z_j \\ 
    &= \sum_i \sum_j z_i \left( \sum_k \phi_k(x^{(i)}) \phi_k(x^{(j)}) \right) z_j \\ 
    &= \sum_k \sum_i \sum_j z_i \phi_k(x^{(i)}) \phi_k(x^{(j)}) z_j \\ 
    &= \sum_k \left( \sum_i z_i \phi_k(x^{(i)}) \right) \left( \sum_j z_j \phi_k(x^{(j)}) \right) \\ 
    &= \sum_k \left( \sum_i z_i \phi_k(x^{(i)}) \right)^2 \\ 
    &\ge 0. 
\end{align*}
倒数第二步利用了 $\sum_{i,j} a_i a_j = (\sum_i a_i)^2$,其中 $a_i = z_i \phi_k(x^{(i)})$。由于 $z$ 是任意的,这表明 $K$ 是半正定的 ($K \ge 0$)。

因此,证明出,如果 $K$ 是一个有效的核(即,它对应于某个特征映射 $\phi$),那么相应的核矩阵 $K \in \mathbb{R}^{n \times n}$ 是对称半正定的。

\subsection*{有效核的充分条件}

更一般地,上面的条件不仅是必要的,而且也是 $K$ 成为有效核(也称为 Mercer 核)的充分条件。以下结果归功于 Mercer\footnote{许多文献以涉及 $L^2$ 函数的稍微更复杂的形式给出 Mercer 定理,但当输入属性取值为 $\mathbb{R}^d$ 时,这里给出的版本是等价的。}。

\noindent\textbf{定理(Mercer):} 设 $K: \mathbb{R}^d \times \mathbb{R}^d \mapsto \mathbb{R}$ 已知。则 $K$ 是一个有效核的充要条件是,对于任意有限集合 $\{x^{(1)}, \dots, x^{(n)}\}$ ($n < \infty$),相应的核矩阵是对称半正定的。

给定一个函数 $K$,除了尝试找到一个与之对应的特征映射 $\phi$ 之外,这个定理因此提供了另一种测试它是否是有效核的方法。在问题集 2 中也将有机会进一步探索这些想法。

课上还简要讨论了几个其他核的例子。例如,考虑手写数字识别问题,给定一个手写数字(0-9)的图像(16x16 像素),需要确定它是哪个数字。使用简单的多项式核 $K(x, z) = (x^T z)^k$ 或高斯核,SVM 能够在此问题上获得非常好的性能。这尤其令人惊讶,因为输入属性 $x$ 仅仅是图像像素值的 256 维向量,系统没有关于视觉的先验知识,甚至没有关于哪些像素相邻的知识。在课堂上简要讨论的另一个例子是,如果试图对字符串进行分类(例如,$x$ 是由氨基酸串成的蛋白质),那么构建一个合理、“小”的特征集对于大多数学习算法来说似乎很困难,特别是如果不同的字符串长度不同。然而,考虑让 $\phi(x)$ 是一个特征向量,它计算 $x$ 中每个长度为 $k$ 的子字符串的出现次数。如果考虑英文字母组成的字符串,那么有 $26^k$ 个这样的字符串。因此,$\phi(x)$ 是一个 $26^k$ 维向量;即使对于中等大小的 $k$,这可能也太大了,无法有效处理(例如,$26^4 \approx 460000$)。但是,使用(类似动态规划的)字符串匹配算法,可以有效地计算 $K(x, z) = \phi(x)^T \phi(z)$,这样就可以在这个 $26^k$ 维特征空间中隐式地工作,而无需显式计算特征向量。



\subsection*{核方法的应用}

已经看到了核方法在线性回归中的应用。在下一部分,将介绍支持向量机,核方法可以直接应用于其中。这里不再赘述。实际上,核方法的思想比线性回归和 SVM 具有更广泛的适用性。具体来说,如果有一个学习算法,可以完全用输入属性向量之间的内积 $\langle x, z \rangle$ 来表达,那么通过将其替换为核 $K(x, z)$(其中 $K$ 是一个核),就可以“神奇地”让算法在对应于 $K$ 的高维特征空间中高效工作。例如,这个核技巧可以应用于感知机,推导出核感知机算法。本课程后面将看到的许多算法也将适用于这种方法,这种方法被称为“核技巧”。